% !TEX root = 6490.tex

\section{Reductive groups}

Let $k$ be an algebraically closed field of characteristic zero. Let 
$G_{/k}$ be a reductive group, i.e.~$\urad G=1$. A good example is $\GL(n)$. 
We could like to determine $G$ up to isomorphism via some combinatorial data. 
Let $T\subset G$ be a maximal torus. Better, let $G_{/k}$ be a split reductive 
group and $k$ arbitrary of characteristic zero. Let $X=\characters^\ast(T)$. 
We have the adjoint representation $\adjoint:G\to \GL(\fg)$. Just as before, we 
can write 
\[
  \fg = \fg_0\oplus \bigoplus_{\alpha\in R} \fg_\alpha ,
\]
where $\fg_\alpha=\{x\in \fg:t x=\alpha(t) x\text{ for all }t\}$ and 
$R=\{\alpha\in X\smallsetminus 0:\fg_\alpha\ne 0\}$. 

Note that $R\subset X_\dR$ need not be a root system, since $R$ may only span a 
proper subspace of $X_\dR$. For example, if $G$ is itself a torus, then 
$R=\varnothing$. However, $R\subset V\subset X_\dR$, where $V=\dR\cdot R$, is a 
root system, and it determines the semisimple group $G/\rad G$ up to isogeny. 
The datum ``$R\subset X$'' is not in general enough to determine $G$. 

Let $\characters_\ast(T)=\hom(\Gm,T)$. There is a perfect pairing (composition) 
$\characters^\ast(T)\times \characters_\ast(T)\to \dZ=\hom(\Gm,\Gm)$. Here 
$\langle \alpha,\beta\rangle$ is the integer $n$ such that 
$\alpha\circ\beta$ is $(-)^n$. From $G$ we'll construct a root datum, which 
will be an ordered quadruple 
$(\characters^\ast(T),\roots(G,T),\characters_\ast(T),\check\roots(G,T))$. 
Before doing this, we'll define root data in general. 





\subsection{Root data}\label{sec:root-data}

The following definitions are from \cite[XXI]{sga3-iii}. A \emph{dual pair} 
is an ordered pair $(X,\check X)$, where $X$ and $\check X$ are finitely 
generated free abelian groups, together with a pairing 
$\langle\cdot,\cdot\rangle:X\times \check X\to \dZ$ that induces an isomorphism 
$\check X\simeq X^\vee=\hom(X,\dZ)$. Let $(X,\check X)$ be a dual pair, and 
suppose we have two elements $\alpha\in X$, $\check\alpha\in \check X$. Define 
the \emph{reflections} $s_\alpha$ and $s_{\check\alpha}$ by 
\begin{align*}
  s_\alpha(x) &= x-\langle x,\check\alpha\rangle\alpha \\
  s_{\check\alpha}(x) &= x-\langle\alpha,x\rangle\check\alpha .
\end{align*}

\begin{definition}
A \emph{root datum} consists of an ordered quadruple $(X,R,\check X,\check R)$, 
such that 
\begin{itemize}
  \item $(X,\check X)$ is a dual pair. 
  \item $R\subset X$ and $\check R\subset \check X$ are finite sets, 
  \item There is a specified mapping $\alpha\mapsto \check\alpha$ from $R$ to 
    $\check R$. 
\end{itemize}
These data are required to satisfy the following conditions:
\begin{enumerate}
  \item For each $\alpha\in R$, $\langle\alpha,\check\alpha\rangle=2$. 
  \item For each $\alpha\in R$, $s_\alpha(R)\subset R$ and 
    $s_{\check\alpha}(\check R)\subset \check R$. 
\end{enumerate}
\end{definition}

It turns out that $\alpha\mapsto \check\alpha$ is a bijection, and that 
$R$ and $\check R$ are closed under negation. 

Let $\sR=(X,R,\check X,\check R)$ be a root datum. Define category of 
root data. 





\subsection{Root datum of a reductive group}

Let $k$ be an algebraically closed field of characteristic zero, $G_{/k}$ a 
connected reductive group, $T\subset G$ a maximal torus, and 
$r=\dim(T)=\rank(G)$ be the \emph{rank} of $G$. Let $\fg=\lie(G)$. The 
adjoint representation $\adjoint:G\to \GL(\fg)$ when restricted to $T$ 
decomposes as 
\[
  \fg = \fg_0\oplus \bigoplus_{\alpha\in R} \fg_\alpha ,
\]
where $\fg_\alpha$ is the $\alpha$-typical component of $\fg$, and 
$R=\{\alpha\in \characters^\ast(T)\smallsetminus 0:\fg_\alpha\ne 0\}$. We 
sometimes write $R=\roots(G,T)$. By [cite source], we have 
$\centralizer_G(T)=T$. The \emph{Weyl group} of $G$ is 
$\weyl(G,T) = \normalizer_G(T)/\centralizer_G(T) = \normalizer_G(T)/T$. 
In [where?], we proved that $\weyl(G,T)$ is a finite group. 

Let $\alpha:T\to \Gm$ be a root. Then $\ker(\alpha)$ is an algebraic 
subgroup of $T$, but it might not be disconnected. Put 
$T_\alpha=(\ker\alpha)^\circ$; his is a torus of dimension $n-1$. Let 
$G_\alpha=\centralizer_G(T_\alpha)$; it turns out that $G_\alpha$ is also 
reductive. Clearly $T\subset G_\alpha$, so $T$ is a maximal torus in 
$G_\alpha$. The inclusion $T_\alpha\subset \zentrum(G_\alpha)$ is 
tautological. In fact, $\zentrum(G_\alpha)^\circ = T_\alpha$. So 
$T/T_\alpha\monic G_\alpha/T_\alpha$ is a maximal torus of dimension $1$. 

So we have a semisimple group $G_\alpha/T_\alpha$ whose maximal torus is 
one-dimensional. Its Dynkin diagram has type $\typeA_1$, so $G_\alpha/T_\alpha$ 
is either $\SL(2)$ or $\PGL(2)$. [Cite SGA for classification of semisimple 
groups of rank one.]

From the embedding $G_\alpha\monic G$, we get an embedding of Lie algebras 
$\lie(G_\alpha)\monic \fg$. It is known [cite?] that $\dim\fg_\alpha=1$ for all 
$\alpha\in R$. It turns out that 
$\lie(G_\alpha)=\ft\oplus \fg_\alpha\oplus \fg_{-\alpha}$. 

The group $G$ is generated $\{G_\alpha:\alpha\in R\}$. Indeed, at 
the level of Lie algebras, $\sum \lie(G_\alpha)=\ft+\sum \fg_\alpha$, which 
we know is all of $\fg$. 

We know that $\lie(G_\alpha)=\ft\oplus \fg_\alpha\oplus \fg_{-\alpha}$. There 
exists a unique connected $U_\alpha\subset G_\alpha$ with 
$\lie(U_\alpha)=\fg_\alpha$. [Certainly $U_\alpha$ is unique if it exists.] 
Since $\fg_\alpha=k$, the group $U_\alpha$ must be either $\Gm$ or $\Ga$. 

\begin{theorem}
There are isomorphisms $u_\alpha:\Ga\iso U_\alpha$ satisfying 
$\adjoint(t)(u_\alpha(x)) = u_\alpha(\alpha(t) x)$ for all $t\in T$ and 
$x\in \Ga$. 
\end{theorem}

The group $G_\alpha$ is generated by $T$, $U_{\pm\alpha}$ by Lie 
algebra considerations. So the group $G$ is generated by $T$ and 
$\{U_\alpha:\alpha\in R\}$. 

\begin{example}[type $\typeA_n$]
Consider $G=\SL(n+1)$. Our maximal torus 
$T=\{\diagonal(t_1,\dots,t_{n+1}):\prod t_i=1\}$. The group 
$\characters^\ast(T)$ is generated by 
$\chi_i(\diagonal(t_1,\dots,t_{n+1}))=t_i$. The roots are 
$R=\{\chi_i-\chi)j:i\ne j\}$. Take, for example, $\alpha=\chi_1-\chi_2$. 

Write $\diagonal(t)=\diagonal(t_1,\dots,t_{n+1})$. We have 
\[
  T_\alpha=\{\diagonal(t):t_1^2 t_2 \dotsm t_{n+1}=1\} .
\]
One can verify $G_\alpha=\centralizer_G(T_\alpha)$ consists of matrices of the 
form 
\[
  \begin{pmatrix} A \\ & t_3 \\ & & \ddots \\ & & & t_{n+1} \end{pmatrix}
\]
for which $A\in \SL(2)$ and $\det(A)\cdot t_3\dotsm t_{n+1}=1$. Thus 
$\lie(G_\alpha)=\ft\oplus \fg_{\chi_1-\chi_2}\oplus \fg_{\chi_2-\chi_1}$. 

It follows that 
\[
  U_\alpha = \begin{pmatrix} 1 & \ast \\ & 1  \\ & & \ddots \\ & & & 1 \end{pmatrix} \qquad U_{-\alpha} = \begin{pmatrix} 1 \\ \ast & 1  \\ & & \ddots \\ & & & 1 \end{pmatrix}
\]
In general, for $\alpha=\chi_i-\chi_j$, we have 
$U_\alpha = 1+\Ga e_{i j}$. Note that indeed, $\SL(n+1)$ is generated by 
$\{U_\alpha:\alpha\in R\}$ and $T$. 
\end{example}

[Exponentiation: dynamic method for getting the $U_\alpha$\ldots]

Back to our general setup $T\subset G_\alpha\subset G$. We can consider the 
``small Weyl group'' $\weyl(G_\alpha,T)\subset \weyl(G,T)$. Since 
$\rank(G_\alpha/\zentrum(G_\alpha)^\circ)=1$, we have 
$\weyl(G_\alpha,T)=\dZ/2$. Let $s_\alpha$ be the unique generator of 
$\weyl(G_\alpha,T)$. It is known that $W=\weyl(G,T)$ is generated by 
$\{s_\alpha:\alpha\in R\}$. 

The group $W$ acts on $\characters^\ast(T)$. Indeed, if $w\in W$, we have 
$w=\dot n$ for some $n\in \normalizer_G(T)$. Define 
$(w\cdot\chi)(t)=\chi(\dot n^{-1} t \dot n)$. 

Recall that $\characters_\ast(T)=\hom(\Gm,T)$. There is a natural pairing 
$\characters^\ast(T)\times \characters_\ast(T)\to \dZ$, for which 
$\langle \alpha,\beta\rangle$ is the unique $n$ such that $\alpha\beta$ is 
$t\mapsto t^n$. That is, $(\alpha\beta)(t) = t^{\langle\alpha,\beta\rangle}$. 
This pairing induces an isomorphism 
$\characters_\ast(T)\simeq \characters^\ast(T)^\vee$. 

\begin{theorem}
Let $\alpha\in R$. Then there exists a unique 
$\check\alpha\in \characters_\ast(T)$ such that 
$s_\alpha(x) = x-\langle x,\check\alpha\rangle\alpha$ for all 
$x\in \characters^\ast(T)$. Moreover, $\langle\alpha,\check\alpha\rangle=2$. 
\end{theorem}

Equivalently, $s_\alpha(\alpha)=-\alpha$. We put 
$\check R=\{\check\alpha:\alpha\in R\}$. The quadruple 
$\sR(G,T) = (\characters^\ast(T),R,\characters_\ast(T),\check R)$ is the 
\emph{root datum} of $G$; it will determine $G$ up to isomorphism (not just 
isogeny). 




