% !TEX root = 6490.tex

\section{Lie algebras}





A \emph{Lie algebra} is a linear object whose representation theory is (in 
principle) manageable. To any linear algebraic group $G$ we will associate a 
Lie algebra $\fg=\lie(G)$, and study the representation theory of $G$ via that 
of $\fg$. 





\subsection{Definition and first properties}

Fix a field $k$. Eventually we'll avoid characteristic $2$, and some theorems 
will only be valid in characteristic zero. 

\begin{definition}
A \emph{Lie algebra} over $k$ is a $k$-vector space $\fg$ equipped with a map 
$[\cdot,\cdot]:\fg\times \fg\to \fg$ such that the following hold:
\begin{itemize}
  \item $[\cdot,\cdot]$ is bilinear (it factors through $\fg\otimes\fg$). 
  \item $[x,x]=0$ for all $x\in \fg$ ($[\cdot,\cdot]$ factors through 
    $\bigwedge^2 \fg$).
  \item The Jacobi identity $[x,[y,z]]+[y,[z,x]]+[z,[x,y]]=0$ holds for all 
    $x,y,z\in \fg$.  
\end{itemize}
\end{definition}

We could have defined a Lie algebra over $k$ as being a $k$-vector space 
$\fg$ together with $[\cdot,\cdot]:\bigwedge^2 \fg\to \fg$ satisfying the 
Jacobi identity. We call $[\cdot,\cdot]$ the \emph{Lie bracket}. 

The Lie bracket satisfies a number of basic properties. For example, 
$[x,y]=-[y,x]$ because 
\begin{align*}
  0 &= [x+y,x+y] && \text{(alternating)} \\
    &= [x,x]+[x,y]+[y,x]+[y,y] && \text{(bilinear)} \\
    &= [x,y] + [y,x] . && \text{(alternating)}
\end{align*}
Later on, we'll use an alternate form of the Jacobi identity: 
\begin{equation*}\label{eq:jac-ident}\tag{$*$}
  [x,[y,z]]-[y,[x,z]] = [[x,y],z] .
\end{equation*}





\subsection{The main examples}

\begin{example}
Let $V$ be any $k$-vector space. Then the zero map $\bigwedge^2 V\to V$ 
trivially satisfies the Jacobi identity. We call any Lie algebra whose 
bracket is identically zero \emph{commutative} (or \emph{abelian}). 
\end{example}

\begin{example}[General linear]
Again, let $V$ be a $k$-vector space. The Lie algebra $\gl(V)=\End_k(V)$ as a 
$k$-vector space, with bracket $[X,Y]=X\circ Y-Y\circ X$. It is a good exercise 
(which everyone should do at least once in their life) to check that this 
actually is Lie algebra. When $V=k^n$, we write $\gl_n(k)$ instead of 
$\gl(k^n)$. 
\end{example}

Often, we will just write $\gl_n$ instead of $\gl_n(k)$. We will also do this 
for the other named Lie algebras. 

\begin{example}[Special linear]
Put $\Sl_n(k)=\{X\in \gl_n(k):\trace X=0\}$. Since $\trace [X,Y]=0$, this is a 
Lie subalgebra of $\gl_n$. In fact, $[\gl_n,\gl_n]\subset \Sl_n$. 
\end{example}

\begin{example}[Special orthogonal]
Put $\so_n(k) = \{X\in \gl_n(k):\transpose X=-X\}$. This is a Lie subalgebra 
of $\gl_n(k)$ because if $X,Y\in \so_n(k)$, then 
\begin{align*}
  \transpose{[X,Y]} 
    &= \transpose Y \transpose X-\transpose X \transpose Y \\
    &= (-Y)(-X) - (-X)(-Y) \\
    &= -[X,Y] .
\end{align*}
\end{example}

\begin{example}[Symplectic]
Let $J_n=\begin{pmatrix} & -I_n \\ I_n \end{pmatrix}\in \gl_{2n}(k)$. Put 
\[
  \fsp_{2n}(k) = \{X\in \gl_{2n}(k):J X+\transpose X\cdot J=0\} .
\]
As an exercise, check that this is a Lie subalgebra of $\gl_{2n}$. 
\end{example}

As an exercise, show that $\dR^3$ with the bracket $[u,v]=u\times v$ (cross 
product) is a Lie algebra over $\dR$. 

\begin{theorem}[Ado]
Any finite dimensional Lie algebra over $k$ is isomorphic to a Lie subalgebra 
of some $\gl_n(k)$. 
\end{theorem}
\begin{proof}
In characteristic zero, this is in \cite[I \S 7.3]{bourbaki-lie-alg-1-3}. The 
positive-characteristic case is dealt with in \cite[VI \S3]{jacobson-1979}.
\end{proof}

Another good exercise is to realize $\dR^3$ with bracket $[u,v]=u\times v$ as 
a subalgebra of some $\gl_n(\dR)$. 





\subsection{Homomorphisms and the adjoint representation}

\begin{definition}
A \emph{homomorphism} of Lie algebras $\varphi:\fg\to \fh$ is a $k$-linear map 
such that $\varphi([x,y])=[\varphi x,\varphi y]$. 
\end{definition}

If $\varphi:\fg\to \fh$ is a homomorphism of Lie algebras, then 
$\fa=\ker(\varphi)$ is a Lie subalgebra of $\fg$. This is easy: 
\begin{align*}
  \varphi[x,y]
    &= [\varphi(x),\varphi(y)] \\
    &= [0,0] \\
    &= 0 .
\end{align*}

\begin{definition}
A Lie subalgebra $\fa$ of $\fg$ is an \emph{ideal} if 
$[\fa,\fg]\subset \fa$, i.e.~$[a,x]\in \fa$ for all $a\in \fa$, $x\in \fg$. 
\end{definition}

Equivalently, $\fa\subset \fg$ is an ideal if $[\fg,\fa]\subset \fa$. If 
$\fa\subset \fg$ is an ideal, we define the \emph{quotient algebra} $\fg/\fa$ 
to be $\fg/\fa$ as a vector space, with bracket induced by that of $\fg$. So 
\[
  [x+\fa,y+\fa] = [x,y]+\fa .
\]
Let's check that this makes sense. If $a_1,a_2\in \fa$, then 
\begin{align*}
  [x+a_1,x+a_2] 
    &= [x,y] + [x,a_2]+[a_1,y]+[a_1,a_2] \\
    &\equiv [x,y] \pmod \fa .
\end{align*}

If $\varphi:\fg\to \fh$ is a Lie homomorphism, then we get an induced 
isomorphism  $\fg/\ker(\varphi)\iso \image(\varphi)\subset \fh$. 

\begin{example}
Give $k$ the trivial Lie bracket. Then $\trace:\gl_n(k) \to k$ is a a 
homomorphism. Indeed, 
\[
  \trace [X,Y]=\trace(X Y)-\trace(Y X)= 0 .
\]
We could have defined $\Sl_n(k)=\ker(\trace)$; this is an ideal in 
$\gl_n$, and $\gl_n(k)/\Sl_n(k)\iso k$. 
\end{example}

\begin{definition}
Let $\fg$ be a Lie algebra over $k$. The \emph{adjoint representation} of $\fg$ 
is the map $\adjoint:\fg\to \gl(\fg)$ defined by $\adjoint(x)(y) = [x,y]$ for 
$x,y\in \fg$. 
\end{definition}

It is easy to check that $\adjoint:\fg\to \gl(\fg)$ is $k$-linear, that is 
$\adjoint(c x)=c\adjoint(x)$ and $\adjoint(x+y)=\adjoint(x)+\adjoint(y)$. 
It is bit less obvious how the adjoint action behaves with respect to 
the Lie bracket. We compute 
\begin{align*}
  \adjoint([x,y])(z) 
    &= [[x,y],z] \\
    &=_\eqref{eq:jac-ident} [x,[y,z]]-[y,[x,z]] \\
    &= (\adjoint(x) \adjoint(y))(z) - (\adjoint(y)\adjoint(x))(z) .
\end{align*}
Thus $\adjoint[x,y] = [\adjoint(x),\adjoint(y)]$ and we have shown that 
$\adjoint:\fg\to \gl(\fg)$ is a homomorphism of Lie algebras. If 
$n=\dim_k(\fg)<\infty$, then $\adjoint:\fg\to \gl(\fg)\simeq \gl_n(k)$ is 
an interesting finite-dimensional representation of $\fg$. If 
$\dim(\fg)>1$, it cannot be surjective, and there are easy ways for it to 
fail to be injective. 

\begin{definition}
Let $\fg$ be a Lie algebra over $k$. The \emph{center} of $\fg$ is 
\[
  \zentrum(\fg) = \{x\in \fg:[x,y]=0\text{ for all }y\in \fg\} .
\]
\end{definition}

Note that $\zentrum(\fg)=\ker(\adjoint)$. There is a natural injection 
$\fg/\zentrum(\fg)\monic \gl(\fg)$. 

\begin{definition}
A Lie algebra $\fg$ is \emph{simple} if it is non-commutative, and has no 
ideals except $0$ and $\fg$. 
\end{definition}

Just as with simple algebraic groups, there is a classification theorem for 
simple Lie algebras. Even better, since Lie algebras are more ``rigid'' than 
algebraic groups, we don't have to worry about isogeny. 

\begin{theorem}
Let $k$ be an algebraically closed field of characteristic zero. Up to 
isomorphism, every Lie algebra is a member of the following list: 
\begin{center}
\begin{tabular}{c|c}
$\typeA_n$ ($n\geqslant 1$) & $\Sl_{n+1}$ \\
$\typeB_n$ ($n\geqslant 2$) & $\so_{2n+1}$ \\
$\typeC_n$ ($n\geqslant 3$) & $\fsp_{2n}$ \\
$\typeD_n$ ($n\geqslant 4$)  & $\so_{2n}$ \\
exceptional & $\fe_6$, $\fe_7$, $\fe_8$, $\ff_4$, $\fg_2$
\end{tabular}
\end{center}
\end{theorem}

The classification theorem for algebraic groups is proved via the 
classification theorem for Lie algebras, which ends up being a matter of 
combinatorics. 





\subsection{Tangent spaces}\label{sec:tangent-space}

For the sake of clarity, we give a scheme-theoretic definition of algebraic 
groups:

\begin{definition}
Let $S$ be a scheme. An \emph{algebraic group} over $S$ is an affine group 
scheme of finite type over $S$. 
\end{definition}

For the moment, we will say that an algebraic group $G$ over $S$ is 
\emph{linear} if $G$ is a subgroup scheme of $\GL(\sL)$ for some locally 
free sheaf $\sL$ on $S$. 

Fix a field $k$, and let $G\subset \GL(n)_{/k}$ be a linear algebraic group cut 
out by polynomials $f_1,\dots,f_r$. For any $k$-algebra $R$, we define
\[
  G(R) = \{g\in \GL_n(R):f_1(g)=\cdots = f_r(g)=0\} .
\]
This is a functor $\algebras k\to \sets$. Since $G$ is a subgroup scheme of 
$\GL(n)_{/k}$, the set $G(R)$ inherits the group structure from $\GL_n(R)$. 
So we will think of $G$ as a functor $G:\algebras k \to \groups$. By the Yoneda 
Lemma, the variety $G$ is determined by its functor of points 
$G:\algebras k\to \sets$. 

We are especially interested in the $k$-algebra of \emph{dual numbers}, 
$k[\varepsilon]=k[x]/x^2$. Informally, $\varepsilon$ should be thought of as a 
``infinitesimal quantity'' in the style of Newton and Leibniz. The scheme 
$\spectrum(k[\varepsilon])$ should be thought of as a ``point together with 
a direction.'' 

\begin{hard}
For the moment, let $k$ be an arbitrary base ring, $X_{/k}$ a scheme. We define 
the \emph{$n$-Jet space} of $X$ to be the scheme $\jet_n X$ whose functor of 
points is $(\jet_n X)(A) = X(A[t]/t^{n+1})$. By \cite{vojta-2007}, this 
functor is representable, so $\jet_n X$ is actually a scheme. The first jet 
space $\jet_1 X$ is called the \emph{tangent space} of $X$, and denoted 
$\tangent X$. The maps $A[t]/t^{n+1}\epic A[t]/t^n$ induce projections 
$\jet_{n+1} X\to \jet_n X$. In particular, the tangent space of $X$ comes with 
a canonical projection $\pi:\tangent X\to X$. 
\end{hard}

Consider $G(k[\varepsilon])$. As a set, this consists of invertible $n\times n$ 
matrices with entries in $k[\varepsilon]$ on which the $f_i$ vanish. The map 
$\varepsilon\mapsto 0$ from $k[\varepsilon]\to k$ induces a group homomorphism 
$\pi:G(k[\varepsilon])\to G(k)$. We will consider this map as the ``tangent 
space'' of $G(k)$. For each $g\in G(k)$, the fiber 
$\pi^{-1}(g)$ is the ``tangent space of $G$ at $g$.'' 

Note that 
\[
  \pi^{-1}(g) = \{g+\varepsilon v:v\in \matrices_n(k):f_i(g+\varepsilon v)=0\text{ for all }i\} .
\]
For each $s$, the polynomial $f_s$ has a Taylor series expansion 
\[
  f_s(x_{i j}) = f_s(g) + \sum_{i,j} \frac{\partial f_s}{\partial x_{i,j}}(g) (x_{i,j}-g_{i,j}) .
\]
Since $f_s(g)=0$, we get 
$f_s(g+\varepsilon v) = \varepsilon \sum_{i,j} \frac{\partial f_s}{\partial x_{i,j}}(g) v_{i,j}$. 
It follows that 
\[
  \pi^{-1}(g) = \left\{g+\varepsilon v:v\in \matrices_n(k):\sum_{i,j} \frac{\partial f_s}{\partial x_{i j}}(g) v_{i j}=0\text{ for all }1\leqslant s\leqslant r\right\} .
\]
We will be especially interested in $\fg=\pi^{-1}(1)$. We will turn this into a 
Lie algebra using the group structure on $G$. 

\begin{hard}
From the scheme-theoretic perspective, the identity element $1\in G(k)$ comes 
from a section $e:\spectrum(k)\to G$ of the structure $G\to \spectrum(k)$. The 
\emph{scheme-theoretic Lie algebra} of $G$ is the fiber product 
\[
  \fg = e^\ast 1 = \tangent(G)\times_G \spectrum(k) .
\]
By definition, $\fg(A)=\ker\left(G(A[\varepsilon])\to G(A)\right)$ for any 
$k$-algebra $A$. Just as with algebraic groups, we will write $\fg_{/k}$ for 
$\fg$ thought of as a group scheme 
over $k$, and just $\fg$ for the $\fg(k)$. 
\end{hard}





\subsection{Functors of points}

Recall that schemes can be thought of as functors $\algebras k\to \sets$. For 
example, affine $n$-space is the functor $\dA^n(A)=A^n$. In fact, this is an 
algebraic group, the operation coming from addition on $A$. Moreover, $\dA^n$ 
is \emph{representable} in the sense that 
\[
  \dA^n(A) = \hom_k(k[x_1,\dots,x_n],A) ,
\]
via $(a_1,\dots,a_n)\mapsto (x_i\mapsto a_i)$. Similarly, 
$\SL(n)_{/k}$ sends a $k$-algebra $A$ to $\SL_n(A)$. It is represented by the 
ring $k[x_{i j}]/(\det(x)-1)$. In general, we take some polynomials 
$f_1,\dots,f_r\in k[x_1,\dots,x_n]$, and consider the subscheme 
$X=V(f_1,\dots,f_r)\subset \dA^n$ which represents the functor 
\[
  X(A) = \{a\in A^n:f_1(a)=\cdots = f_r(a)=0\} .
\]
This has coordinate ring $k[x_1,\dots,x_n]/(f_1,\dots,f_r)$. Note that 
\emph{any} finitely generated $k$-algebras is of this form. In other words, all 
affine schemes of finite type over $k$ are subschemes of some affine space. 

Clasically, one gives $X(\bar k)$ the \emph{Zariski topology}. If 
$f\in k[X]=\dA^1(X)$, we get (by definition) a function 
$f:X(\bar k)\to \bar k$. We define open subsets of $X(\bar k)$ to be 
those of the form $\{x\in X(\bar k):f(x)\ne 0\}$. More scheme-theoretically, 
we say that an \emph{open subscheme} of $X$ is the complement of any 
$V(f)$ for $f\in k[X]$. 

\begin{hard}
We can think of the Zariski topology as giving a Grothendieck topology on the 
category of affine schemes over $k$. It is subcanonical, i.e.~all representable 
functors are sheaves. So if we formally put $\affine k = (\algebras k)^\circ$, 
then the category of schemes over $k$ embeds into 
$\sheaves_\zariski(\affine k)$. When working with algebraic groups, often it is 
better with the \'etale or fppf topologies. In particular, we will think of 
quotients as living in $\sheaves_\fppf(\affine k)$. 
\end{hard}

\begin{definition}
Let $k$ be a feld. A \emph{variety} over $k$ is a reduced separated scheme of 
finite type over $k$. 
\end{definition}

In particular, an affine variety is determined by a reduced $k$-algebra. We 
can directly associate a ``geometric object'' to a $k$-algebra $A$, namely its 
\emph{spectrum} $\spectrum(A)=\{\fp\subset A\text{ prime ideal}\}$. Finally, we 
note that if $X=\spectrum(A)$ is of finite type over $k$ and if $k=\bar k$, then 
$X(k)=\hom_k(A,k)$ is the set of maximal ideas in $A$. 





\subsection{Affine group schemes and Hopf algebras}

A affine group $G_{/k}$ will give us a functor $G:\algebras k\to \groups$. As 
such, it should have morphisms 
\begin{align*}
  m :G\times G &\to G && \text{``multiplication''} \\
  e:1=\spectrum(k) &\to G && \text{``identity''} \\
  i:G &\to G && \text{``inverse''}
\end{align*}
These correspond to ring homomorphisms 
\begin{align*}
  \Delta:A &\to A\otimes_k A && \text{``comultiplication''} \\
  \varepsilon : A &\to k && \text{``counit''} \\
  \sigma:A &\to A && \text{``coinverse''}
\end{align*}
Obviously the maps on groups satisfy some axioms like associativity etc. These 
can be phrased by the commutativity of diagrams like 
\[\begin{tikzcd}
  G\times G\times G \ar[r, "\id\times m"] \ar[d, "m\times \id"] 
    & G\times G \ar[d, "m"] \\ 
  G\times G \ar[r, "m"] 
    & G .
\end{tikzcd}\]
Most sources do not give all these diagrams explicitly. One that does is 
the book \cite{waterhouse-1979}. 

It is a good exercise to show that we have the following maps for the additive 
and multiplicative groups:
\begin{center}
\begin{tabular}{c|cccc}
group & ring & comultiplication & coident. & coinverse \\ \hline
$\Ga$ & $k[x]$ & $x\mapsto x\otimes 1+1\otimes x$ & $x\mapsto 0$ & $x\mapsto -x$ \\
$\Gm$ & $k[x^{\pm 1}]$ & $x\mapsto x\otimes x$ & $x\mapsto 1$ & $x\mapsto x^{-1}$
\end{tabular}
\end{center}
The analogue of associativity for coordinate rings and comultiplication is the 
commutativity of the following diagram:
\[\begin{tikzcd}
  A \ar[r, "\Delta"] \ar[d, "\Delta"] 
    & A\otimes_k A \ar[d, "\id\otimes \Delta"] \\
  A\otimes_k A \ar[r, "\Delta\otimes \id"] 
    & A\otimes_k A\otimes_k A
\end{tikzcd}\]
The tuple $(A,\Delta,\varepsilon,\sigma)$ is called a \emph{Hopf algebra}. In 
principle, all theorems about algebraic groups can be rephrased as theorems 
about Hopf algebras, but this is not very illuminating. The one advantage is 
that it is possible to speak of Hopf algebras for which the ring $A$ is 
\emph{not} commutative. These show up in algebraic number theory, combinatorics, 
and physics. 

\begin{theorem}\label{thm:groups-linear}
If $G$ is an affine group scheme of finite type over $k$, then it is a 
linear algebraic group over $k$. 
\end{theorem}
\begin{proof}
This is \cite[6b.11.11]{sga3-i}. Essentially, one looks at the action of $G$ 
on the (possibly infinite-dimensional) vector space $k[G]$. A general theorem 
yields a faithful finite-dimensional representation $V$, and we get $G$ as a 
subgroup-scheme of $\GL(V)$. 
\end{proof}

\begin{hard}
If $S$ is a dedekind scheme (noetherian, one-dimensional, regular) and $G_{/S}$ 
is a affine flat separated group scheme of finite type over $S$, then the 
obvious generalization of autoref{thm:groups-linear} holds for $G$; it is a 
closed subgroup of $\GL(\sL)$ for $\sL$ a locally free sheaf on $S$ 
\cite[6b.11.11.1]{sga3-i}. 
Interestingly, this fails for more general base schemes. 
\end{hard}





\subsection{Examples of group schemes}

So all group varieties over $k$ embed into $\GL(n)_{/k}$ \emph{as an algebraic 
group}. We can think of an algebraic group in several ways: 
\begin{enumerate}
  \item As a group-valued functor on $\algebras k$. 
  \item As a commutative Hopf algebra over $k$. 
  \item As a subgroup of $\GL_n(k)$ cut out by poynomials. 
\end{enumerate}

\begin{example}
The group of \emph{$n$-th roots of unity} is $\dmu_n$. Its coordinate ring is 
$k[x]/(x^n-1)$. So 
\[
  \dmu_n(A) = \{a\in A:a^n=1\} ,
\]
with the group operation coming from multiplication in $A$. Note that 
$\dmu_n\subset \Gm$. If the base field $k$ has characteristic 
$p\nmid n$, then $\dmu_n(\bar k)$ is a cyclic group of order $n$. However, if 
$n=p$, then $x^p-1=(x-1)^p$. So the coordinate ring 
$k[x]/((x-1)^p)$ is non-reduced. This is our first example of a group scheme 
that is \emph{not} a group variety. One realization of this is that 
$\dmu_p(\overline{\dF_p})=1$. These problems can only occur in characteristic 
$p>0$. If $G$ is a group scheme over a characteristic zero field, then 
$G$ is automatically reduced. 

Let $\phi:\Gm\to \Gm$ be the homomorphism $x\mapsto x^p$. Note that 
$\ker(\phi)=\dmu_p$. On $\overline{\dF_p}$-points, this map is an isomorphism, 
but $\phi$ is \emph{not} an isomorphism of algebraic groups. 
\end{example}

Let $G_{/k}$ be a linear algebraic group. Recall that we defined the \emph{Lie 
algebra} $\fg=\lie(G)$ of $G$ to be the functor on $k$-algebras given by 
\[
  \fg(A) = \ker\left(G(A[\varepsilon]/\varepsilon^2) \to G(A)\right) .
\]
In \autoref{sec:tangent-space}, we saw that if $G\subset \GL(n)_{/k}$ is cut out by 
$f_1,\dots,f_r$, then the $k$-valued points could be computed as 
\[
  \fg = \left\{X\in \matrices_n(k):\sum_{i,j} \frac{\partial f_\alpha}{\partial x_{i j}}(1_n)\cdot X_{i j}=0\right\} .
\]

\begin{example}[Special linear]
Let $G=\SL(n)_{/k}$. Note that $G\subset \GL(n)$ is cut out by $\det=1$. So 
$\fg=\lie(G)=\Sl_n$ is 
\[
  \Sl_n = \left\{1+X\varepsilon:X\in \matrices_n(k)\text{ and }\det(1+X\varepsilon)=1\right\} .
\]
The expansion of determinant is 
$\det(1+X\varepsilon)=1+\trace(X)\varepsilon + O(\varepsilon^2)$. Thus 
\[
  \Sl_n = \left\{X\in \matrices_n(k):\trace(X)=0\right\} .
\]
As a functor, $\Sl_n(A)=\{X\in \matrices_n(A):\trace X=0\}$. 
\end{example}

\begin{example}[Orthogonal]
Recall that $G=\Or(n)\subset \GL(n)$ is cut out by $g \transpose g=1$. So 
\begin{align*}
  \fg
    &= \{1+X\varepsilon  (1+X\varepsilon) \transpose{(1+X\varepsilon)}=1\} \\
    &= \{1+X\varepsilon:X+\transpose X=0\} \\
    &\simeq \{X\in \matrices_n(k):\transpose X=-X\} \\
    &= \so_n(k) .
\end{align*}
If the characteristic of $k$ is not $2$, then $\dim(\so_n)=n(n-1)/2$. 
\end{example}





\subsection{Lie algebra of an algebraic group}

Fix a field $k$. Let $G_{/k}$ be a linear algebraic group. For any $k$-algebra 
$A$, write $A[\varepsilon]=k[t]/t^2$. Then $A\mapsto G(A[\varepsilon])$ is a 
new group functor, and we can define the \emph{Lie algebra} of $G$ to be 
\[
  \lie(G)(A) = \ker\left(G(A[\varepsilon]) \to G(A)\right) .
\]
A priori, this is a group functor. Often, we will write $\fg=\lie(G)$ to mean 
$\lie(G)(k)$. If $G\subset \GL(n)$ is cut out by $f_1,\dots,f_r$, then 
\[
  \fg=\left\{1+\varepsilon X:X\in \matrices_n(k)\text{ and }f_\alpha(1+\varepsilon X)=0\text{ for all }\alpha\right\} .
\]
The group operation is addition on $X$, 
i.e.~$(1+\varepsilon X)(1+\varepsilon Y)=1+\varepsilon(X+Y)$. Thus 
$\fg$ is a $k$-vector space. 

\begin{hard}
This section will contain many ``high-level'' interludes on the functorial 
definition of $\lie(G)$. They are all strongly influenced by the exposition in 
\cite[II \S3--4]{sga3-i}. Let $\sO:\algebras k\to \algebras k$ be the functor 
$A\mapsto A$. Note that $\sO$ is represented by $k[t]$. Thus $\dA^1_{/k}$ has 
the structure not only of a group scheme, but of a $k$-ring scheme, or 
$(k,k)$-biring in the sense of \cite{borger-wieland-2005}. We define an 
\emph{$\sO$-module} to be a functor $V:\algebras k\to \sets$ such that for each 
$A$, the set $V(A)$ is given an $\sO(A)=A$-module structure in a functorial 
way. 

If $S:\algebras k \to \sets$ is a functor, we will write 
$\schemes S$ for the category of representable functors 
$X:\algebras k\to \sets$ together with a morphism $X\to S$. A morphism 
$(X\to S)\to (Y\to S)$ in $\schemes S$ is a morphism of functors 
$X\to Y$ making the following diagram commute:
\[\begin{tikzcd}
  X \ar[r] \ar[dr] 
    & Y \ar[d] \\
  & S 
\end{tikzcd}\]
There is an easy generalization of $\sO$ to a ring object 
$\sO\in\schemes S$. It sends a scheme $X$ to the ring $\h^0(X,\sO_X)$ of 
regular functions on $X$. 

For a scheme $X$, the first jet space 
$\jet_1 X$ is naturally an $\sO$-module in $\schemes X$. We could be fancy and 
say that for any ring $A$, $A[\varepsilon]$ is naturally an abelian group 
object in $\algebras A/A$, or we could define the $\sO$-module structure on 
$\jet_1 X\xrightarrow\pi X$ directly. For $\varphi:\spectrum(A)\to X$, we need 
to give 
\[
  (\jet_1 X)_{/X}(A) = \{f\in (\jet_1 X)(A):\pi\circ f = \varphi\} 
\]
the structure of an $A$-module. Given $a\in A$, define 
$\sigma_a:A[\varepsilon]\to A[\varepsilon]$ by 
$\varepsilon\mapsto a\varepsilon$. This is a ring homomorphism,so it induces 
$\sigma_a:\jet_1 X(A)\to \jet_1 X(A)$. The restriction of this map to 
$(\jet_1 X)_{/X}(A)$ is the desired action of $A$. See \cite[II 3.4.1]{sga3-i} 
for a more careful proof that this gives $(\jet_1 X)_{/X}$ the structure of 
an $\sO$-module in $\schemes X$. 

The point of all this is that if $M$ is an $\sO$-module in 
$\schemes Y$ and $f:X\to Y$ is a morphism of schemes, then 
$f^\ast M$, defined by 
\[
  (f^\ast M)(\spectrum(A)\xrightarrow p X) = M(\spectrum(A)\xrightarrow p X\xrightarrow f Y)
\]
is naturally an $\sO$-module in $\schemes X$. So if $G$ is an algebraic group, 
$\jet_1 X$ is naturally an $\sO$-module in $\schemes G$, and the morphism 
$e:1\to G$ gives $\fg=e^\ast \jet_1 G$ the structure of an $\sO$-module in 
$\schemes k$. 
\end{hard}

\begin{example}[Symplectic]
Let $J=\begin{pmatrix} & -1_n \\ 1_n \end{pmatrix}$. Recall that 
\[
  \Sp_{2n}(A) = \left\{g\in \GL_{2n}(A) : \transpose g J g=J\right\} .
\]
Let's compute the Lie algebra of $\Sp_{2n}$. For $X\in \matrices_{2n}(A)$, we 
have 
\begin{align*}
  \transpose{(1+\varepsilon X)} J (1+\varepsilon X) 
    &= (1+\varepsilon\transpose X) J (1+\varepsilon X) \\
    &= J+\varepsilon(\transpose X J+J X) .
\end{align*}
So 
$\fsp_{2n}(A) \simeq \left\{X\in \matrices_{2n}(A):\transpose X J+J X = 0\right\}$. 
\end{example}

Algebraic groups can be highly non-abelian, and so far all we've done is give 
$\fg=\lie(G)$ the structure of a vector space. We'd like to give $\fg$ the 
structure of a Lie algebra. 

Let $f:G\to H$ be a homomorphism of algebraic groups. Then $f:G(A)\to H(A)$ is 
a homomorphism for all $k$-algebras $A$. A morphism $f:G\to H$ corresponds 
by the Yoneda Lemma to a unique element $\varphi\in \hom_k(k[H],k[G])$. For all 
$A$, the induced map $G(A)\to H(A)$ is defined by 
$\psi\mapsto \psi\circ\varphi$ via the identifications 
$G(A)=\hom(k[G],A)$ and $H(A)=\hom(k[H],A)$. 

\begin{example}
Recall that $\Gm(A)=A^\times = \hom(k[x^{\pm 1}], A)$. All homomorphisms 
$f:\Gm\to \Gm$ come from $x\mapsto x^n$ for some $n\in \dZ$. On functors of 
points, this is $a\mapsto a^n$ for $a\in A^\times$. 
\end{example}

As before, let $f:G\to H$ be a morphism of algebraic groups. Let 
$\fg=\lie(G)$, $\fh=\lie(G)$. There is an induced map 
$f_\ast=\lie(f):\fg\to \fh$. On functors of points, it is induced by the 
inclusions $\fg\subset \jet_1 G$, $\fh\subset \jet_1 H$ and the fact that 
$f:\jet_1 G\to \jet_1 H$ is a group homomorphism. That is, for each 
$k$-algebra $A$, $f_\ast:\fg(A)\to \fh(A)$ is induced by the commutative 
diagram 
\[\begin{tikzcd}
  G(A[\varepsilon]) \ar[r, "f"] 
    & H(A[\varepsilon]) \\
  \fg(A) \ar[r, "f_\ast"] \ar[u, hook] 
    & \fh(A). \ar[u, hook]
\end{tikzcd}\]
In particular, a representation $\rho:G\to \GL(n)$ induces a map on 
Lie algebras $\fg\to \gl_n$. 

\begin{hard}
The natural way of stating the above is that $\lie$ is a functor from group 
schemes over $k$ to $\sO$-modules in $\schemes k$. What's more, if $G$ is a 
group scheme and $\fg=\lie(G)$, then by \cite[II 3.3]{sga3-i}, 
there is a natural isomorphism of $A$-modules 
$\fg(k)\otimes_k A\iso \fg(A)$. So the functor $\fg$ is actually determined 
by the vector space $\fg(k)$. This justifies our passing between 
$\fg$ and $\fg(k)$. 
\end{hard}

Let's construct the Lie bracket on the Lie algebra of an algebraic group. 
Recall that for a Lie algebra $\fg$, we defined a map 
$\adjoint:\fg\to \gl(\fg) = \End_k(\fg)$ by $\adjoint(x)(y)=[x,y]$. The 
bracket on $\fg$ is clearly determined by the adjoint map. 

For $\fg=\lie(G)$, we'll define the adjoint map directly, then use this to 
give $\fg$ the structure of a Lie algebra. For any $g\in G(A)$, we have an 
action $\adjoint(g):G_A\to G_A$ given by $x\mapsto g x g^{-1}$ for any 
$B\in \algebras A$. So we have a homomorphism of group functors 
$\adjoint:G\to \automorphisms(G)$. In particular, $G(k)$ acts on $G$ by 
$k$-automorphisms. 

\begin{hard}
The general philosophy, due to Grothendieck and worked out carefully in 
SGA, is that ``everything is a functor.'' That is, we think of every object 
as living in the category of functors $\algebras k\to \sets$. For example, 
if $X$ and $Y$ are schemes over $k$, then $\hom(X,Y)$ is the functor whose 
\[
  \hom(X,Y)(A) = \hom_{\schemes A}(X_A, Y_A) .
\]
We recover the usual hom-set as $\hom(X,Y)(k)$. Similarly, we define 
$\automorphisms(G)(A)=\automorphisms_{\groups_A}(G_A)$. These functors are not 
usually representable, but they are sheaves all the reasonable Grothendieck 
topologies on $\schemes k$. See \cite[I \S 1--3, II \S 1]{sga3-i} for a careful 
exposition along these lines. 
\end{hard}

By functoriality, there is a natural homomorphism 
$\automorphisms(G)\to \automorphisms(\fg)$. Composing this with the map 
$\adjoint:G\to \automorphisms(G)$ gives a map 
$\adjoint:G\to \automorphisms(\fg)$. Explicitly, the map 
$\adjoint:G(A)\to \automorphisms(\fg)(A)=\automorphisms(\fg_A)$ sends 
$g$ to the automorphism $X\mapsto g X g^{-1}$ for any 
$X\in G(B[\varepsilon])$, $B\in \algebras A$. It is easy to see that this 
action respects $\sO$-module structure on $\fg$, so we in fact have a 
representation $\adjoint:G\to \GL(\fg)$. Since 
$\fg(A)=\fg(k)\otimes_k A$, we can think of $\fg$ as an honest 
$k$-vector space and define $\GL(\fg)$ as the functor 
$A\mapsto \automorphisms_A(\fg\otimes_k A)$. In either case, we have a 
morphism $\adjoint:G\to \GL(\fg)$ of linear algebraic groups. 

Apply $\lie(-)$ to the adjoint representation $\adjoint:G\to \GL(\fg)$. We 
get a map $\adjoint:\fg\to \lie(\GL(\fg))=\gl(\fg)$. We claim that 
$\fg$ together with $[x,y]=\adjoint(x)(y)$ is a Lie algebra. 

\begin{example}
Let $G=\GL(n)$. We wish to verify that our functorial definition of 
$\adjoint:\gl_n\to \gl(\gl_n)$ agrees with the elementary definition 
$\adjoint(x)(y)=[x,y]$. Start with the action of $\GL(n)$ on 
$\gl_n$. For any $k$-algebra $A$, this is the action by conjugation of 
$\GL_n(A)\subset \GL_n(A[\varepsilon])$ on 
$\gl_n(A)=\ker(\GL_n(A[\varepsilon])\to \GL_n(A))$. This is because 
\[
  g\cdot (1+\varepsilon x)(g^{-1}) = 1+\varepsilon \adjoint(g)(x) .
\]
So the functorial and elementary definitions of $\adjoint:\GL(n)\to \GL(\gl_n)$ 
agree. 

Now let's differentiate $\adjoint:\GL(n)\to \GL(\gl_n)$. For 
$A\in \algebras k$, we have a commutative diagram:
\[\begin{tikzcd}
  \gl_n(A) \ar[r, hook] \ar[d, "\adjoint"] 
    & \GL_n(A[\varepsilon]) \ar[r, twoheadrightarrow] \ar[d, "\adjoint"]
    & \GL_n(A) \ar[d, "\adjoint"] \\
  \gl(\gl_n)(A) \ar[r, hook]
    & \GL(\gl_n\otimes A[\varepsilon]) \ar[r, twoheadrightarrow] 
    & \GL(\gl_n\otimes A)
\end{tikzcd}\]
All that remains is the simple computation 
\begin{align*}
  \adjoint(1+ \varepsilon x)(y)
    &= (1+\varepsilon x)y(1-\varepsilon x) \\
    &= y+[x,y]\varepsilon
\end{align*}
So as an endomorphism of $\gl_n\otimes A[\varepsilon]$, 
$\adjoint(1+\varepsilon x)$ is of the form $1+\varepsilon [x,-]$. So when we 
identify $1+\varepsilon x$ with $x$, we get that $\adjoint(x)=[x,-]$ as 
desired. See \cite[II 4.8]{sga3-i} for a proof in much greater generality. 
\end{example}

This example yields the general fact that $\lie(G)$ is a Lie algebra. Choose an 
embedding $G\monic \GL(n)$. By construction, the Lie functor is 
limit-preserving, so $\fg\monic \gl_n$. The bracket on $\fg$ (as a subspace of 
$\gl_n$ is induced by the bracket on $\gl_n$. We have seen that the functorial 
definition of the bracket on $\gl_n$ matches the elementary definition, so 
$\gl_n$ is a Lie algebra and $\fg$ is a Lie subalgebras of $\gl_n$. 

Does $\lie(G)$ determine $G$? The answer is an easy no! For example, 
$\SO(n)$ and $\Or(n)$ both have Lie algebra $\so_n$. Also 
$\SL(n)$ and $\PSL(n)$ both have Lie algebra $\Sl_n$. There is are the 
easy example $\lie(\Ga)=\lie(\Gm)=\gl_1$. Finally, 
$\lie(G)=\lie(G^\circ)$, so $\lie(G)$ does not depend on the group 
$\pi_0(G)=G/G^\circ$ of connected components of $G$. However, given a 
semisimple Lie algebra $\fg$, there is a unique split connected, 
simply-connected semisimple algebraic group $G$ with $\lie(G)=\fg$. 

\begin{example}
Let's go over the Lie bracket on $\GL(n)$ in a more elementary manner. We have 
\[
  \gl_n = \ker\left(\GL_n(k[\varepsilon])\to\GL_n(k)\right) = \{1+\varepsilon X:X\in \matrices_n(k)\} \iso \matrices_n(k) .
\]
We'll write elements of $\matrices_n(k[\varepsilon])$ as $x+y\varepsilon$; 
these should be thought of as a ``point'' $x$ together with an ``infinitesimal 
direction'' $y$. The adjoint map is the homomorphism 
$\adjoint:\GL(n)\to \GL(\gl_n)$, defined on $A$-points as the map 
$\GL_n(A)\to \GL(\gl_n(A))$ given by $\adjoint(g)(x) = g x g^{-1}$. So for 
$A=k[\varepsilon]$, the group $\GL_n(k[\varepsilon])$ acts on 
$\matrices_n(k[\varepsilon])$ by conjugation. Take any 
$1+\varepsilon b\in \GL_n(k[\varepsilon])$, and any 
$x+\varepsilon y\in \matrices_n(k[\varepsilon])$. We compute: 
\begin{align*}
  (1+\varepsilon b)(x+\varepsilon y)(1-\varepsilon b)^{-1} 
    &= (x+\varepsilon y+\varepsilon b x)(1-\varepsilon b) \\
    &= x+\varepsilon y + \varepsilon(b x-x b) \\
    &= x+\varepsilon y + \varepsilon [b,x] \\
    &= (1+\varepsilon[b,-])(x+\varepsilon y) .
\end{align*}
Thus $\adjoint(1+\varepsilon b)=1+\varepsilon [b,-]$ as an element of 
$\End_k(\gl_n) = \gl(\gl_n)$. Summing it all up, we have a commutative 
diagram: 
\[\begin{tikzcd}
  \gl_n \ar[r, "\adjoint"] \ar[d, "\wr"]
    & \gl(\gl_n) \ar[d, "\wr"] \\
  \matrices_n \ar[r, "\adjoint"] 
    & \End(\matrices_n)
\end{tikzcd}\]
so that $\adjoint(x)(y) = [x,y]$. 
\end{example}

As we have already seen, $\lie(G)$ does not determine $G$. Even worse, the 
functor $\lie$ is neither full nor faithful. It's not faithful because, for 
example, if $\Gamma$ is a finite group and $G$ a connected group, then 
$\automorphisms(\Gamma)\hookrightarrow \automorphisms(\Gamma\times G)$, but 
since $(G\times\Gamma)^\circ\subset G$, any automorphism of $\Gamma$ acts 
trivially on $\lie(\Gamma\times G)$. To see that $\lie$ is not faithful, 
note that $\hom(\gl_1,\gl_1)$ is a one-dimensional vector space, while 
$\hom(\Gm,\Gm)=\dZ$. 

\begin{theorem}\label{thm:subgroup-lie-alg}
Let $k$ be a field of characteristic zero, $G_{/k}$ a linear algebraic group. 
Let $\fg=\lie(G)$. The map $H\mapsto \lie(H)$ from connected closed subgroups 
of $G$ to Lie subalgebras of $\fg$ is injective and preserves inclusions. 
\end{theorem}
\begin{proof}
If $H_1\subset H_2$, use the functoriality of forming Lie algebras to see that 
$\lie(H_1)\subset \lie(H_2)$ as subalgebras of $\fg$. The nontrivial part is to 
show that $\lie(H)$, as a subalgebra of $\fg$, determines $H$. 
Let $H_1$, $H_2$ be be two closed connected subgroups of $G$ such that 
$\fh=\lie(H_1)=\lie(H_2)$. Then $H_3=(H_1\cap H_2)^\circ$ is a closed connected 
subgroup of $G$. Since fiber products are computed pointwise and the Lie 
functor is exact, we have $\lie(H_3)=\fh$. By \autoref{thm:cartier}, the 
$H_i$ are all smooth, so $\dim(H_i)=\dim \fh$. But the only way 
$H_3\subset H_1$ can both be smooth and irreducible with the same dimension is 
for $H_1=H_3$. Similarly $H_2=H_3$, so $H_1=H_2$. 
\end{proof}

\begin{example}
The subgroup 
$\Gm\simeq \left\{\begin{pmatrix} a \\ & a^{-1} \end{pmatrix}\right\}$ 
and $\Ga\simeq \begin{pmatrix} 1 & \ast \\ & 1 \end{pmatrix}$ of $\GL(2)$ have 
Lie algebras that are abstractly isomorphic, but distinct as subalgebras of 
$\gl_2$. Thus $\lie(H)$, seen as an abstract Lie algebra, does not determine 
$H$ as a subgroup of $G$. 
\end{example}

\autoref{thm:subgroup-lie-alg} fails in positive characteristic. Also, 
surjectivity need not hold, see e.g.~$\GL(5)$. If we instead work in the 
category of smooth manifolds and Lie groups, then \emph{every} Lie subalgebra 
comes from a subgroup. 

\begin{theorem}[Cartier]\label{thm:cartier}
Let $k$ be a field of characteristic zero, $G_{/k}$ a linear algebraic group. 
Then $G$ is smooth. 
\end{theorem}
\begin{proof}
The idea is that for $k=\dC$, $G(\dC)$ must be smooth at ``most'' points 
$g$. But any two points in $G(\dC)$ have diffeomorphic neighborhoods, so 
$G(\dC)$ itself is smooth. For a careful proof in much greater generality, 
see \cite[VI\textsubscript{B} 1.6]{sga3-i}. Even SGA assumes that $G$ is 
locally of finite type, 
but this is not necessary. See \url{http://mathoverflow.net/questions/22553} 
for some discussion and references. 
\end{proof}

\begin{example}
Consider $\dmu_p=\ker(\Gm\xrightarrow p \Gm)$ over $k=\dF_p$. That is, 
$\dmu_p(A)=\{a\in A:a^p=1\}$. The coordinate ring of $\dmu_p$ is 
$k[x]/(x^p-1)$. Note that  
\begin{align*}
  \lie(\dmu_p)(A)
    &= \{1+\varepsilon x:x\in A\text{ and }(1+\varepsilon x)^p=1\} \\
    &= \{1+\varepsilon x:1+(\varepsilon x)^p = 1\} \\
    &= A .
\end{align*}
In other words, $\dmu_p\subset \GL(1)$ has Lie algebra $\gl_1$. Since $\dmu_p$ 
is connected over $\dF_p$, this provides a counterexample to 
\autoref{thm:subgroup-lie-alg} in characteristic $p$. Moreover, 
$\dmu_p(\overline{\dF_p})=1$, so ``$\dmu_p$ has no points.''
\end{example}

If, on the other hand, the base field has 
characteristic not $p$, then $(1+\varepsilon x)^p = 1+p\varepsilon x$, which is 
$1$ exactly when $x=0$. So $\lie(\dmu_p)=0$ as expected. 




