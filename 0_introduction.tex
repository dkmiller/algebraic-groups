% !TEX root = 6490.tex

\section{Introduction}





The latest version of these notes can be found online at 
\url{http://www.math.cornell.edu/~dkmiller/bin/6490.pdf}. The source code 
is at \url{https://github.com/dkmiller/6490}. Comments and corrections are 
appreciated. 





\subsection{Disclaimer}

These notes originated in the course MATH 6490: Linear algebraic groups and 
their Lie algebras, taught by David Zywina at Cornell University. However, the 
notes have been substantially modified since then, and are not an exact 
reflection of the content and style of the original lectures. 





\subsection{Notational conventions}

We follow Bourbaki in writing $\dN,\dZ,\dQ$\ldots for the natural numbers, 
integers, rationals, \ldots. The natural numbers are $\dN=\{1,2,\ldots\}$. 

If $A$ is a commutative ring, $M$ is an $A$-module, and $a\in A$, 
we write $M/a$ for $M/(a M)$. In particular, $A/a$ is the quotient of $A$ by 
the ideal generated by $a$. All abelian groups will be treated as 
$\dZ$-modules. So $A$ is an abelian group and $n\in \dZ$, the quotient 
$A/n$ means $A/n\cdot A$ even if $A$ is written multiplicatively. 

The notation $X_{/S}$ will mean ``$X$ is a scheme over $S$.'' If 
$S=\spectrum(A)$, we will write $X_{/A}$ to mean that $X$ is a scheme over 
$\spectrum(A)$. 

We write $\transpose X$ for the transpose of a matrix $X$. 





\subsection{Main references}

The standard texts are \cite{borel-1991,humphreys-1975,springer-2009}. In 
these, the algebraic geometry is done from scratch in an archaic language. A 
good reference for modern (scheme-theoretic) algebraic geometry is 
\cite{hartshorne-1977}, and a (very abstract) modern reference is the 
three volumes on group schemes \cite{sga3-i,sga3-ii,sga3-iii} from the 
\emph{S\'eminaire de G\'eom\'etrie Alg\'ebrique}. 





\subsection{A bestiary of examples}

Let $k$ be an algebraically closed field, for example $\dC$. Our initial 
working definition of a linear algebraic group is a subgroup $G$ of 
$\GL_n(k)$ defined by polynomial equations. Here are some standard examples:
\begin{itemize}
  \item $\GL_n(k)$ 
  \item the \emph{special linear group} $\SL_n(k)=\{\det=1\}\subset \GL_n(k)$
  \item an \emph{orthogonal group} 
    $\operatorname{O}_n(k)=\{g\in \GL_n(k):g \transpose g=1\}$. This is cut 
    out by the equations 
    \[
      \sum_{j=1}^n a_{i j} a_{k j} = \delta_{i k} .
    \]
  \item the \emph{special orthogonal group} $\SO_n(k)=\operatorname{O}_n(k)\cap \SL_n(k)$
  \item Symplectic groups. Let $J=\begin{pmatrix} & I_n \\ -I_n \end{pmatrix}\in \GL_{2n}(k)$. We put $\Sp_{2n}(k)=\{g\in \GL_{2n}(k):\transpose g J g=J\}$. 
  \item The group of unipotent matrices 
    \[
      U_n(k) = \begin{pmatrix} 1 & \cdots & \ast \\ & \ddots & \vdots \\ & & 1 \end{pmatrix} \subset \GL_n(k)
    \]
    Recall that $g\in \GL_n(k)$ is \emph{unipotent} if $(g-I)^m=0$ for some 
    $m\geqslant 1$. 
  \item We write $\Gm(k)=k^\times$ for the \emph{multiplicative group} 
    $\GL_1(k)$. 
  \item The \emph{additive group} $\Ga(k)=k$, with the usual additive structure. 
    This is actually $U_2(k)$. Indeed, $\Ga\iso U_2$ via 
    $x\mapsto \begin{pmatrix} 1 & x \\ & 1 \end{pmatrix}$. Since 
    \[
      \begin{pmatrix} 1 & x \\ & 1 \end{pmatrix} \begin{pmatrix} 1 & y \\ & 1 \end{pmatrix} = \begin{pmatrix} 1 & x+y \\ & 1 \end{pmatrix}
    \]
    this is actually a morphism of algebraic groups. 
  \item For any $n\geqslant 0$, $\Ga^n(k)=k^n$ with the usual addition is a 
    linear algebraic group. We could embed it into $\GL_{2n}(k)$ via $2\times 2$ blocks 
    and the isomorphism $\Ga\iso U_2$ above.
  \item For any $n\geqslant 1$ we have a \emph{torus} of rank $n$, namely 
    \[
      T(k) = \begin{pmatrix} \ast \\ & \ddots \\ & & \ast \end{pmatrix}\subset \GL_n(k) .
    \]
    This is clearly isomorphic to $\Gm^n(k)$.  
\end{itemize}

This list almost exhausts the class of \emph{simple algebraic groups} over an 
algebraically closed field. But note that all of these groups make sense over 
an arbitrary field. 

The standard references \cite{borel-1991,humphreys-1975,springer-2009} all treat 
algebraic groups in terms of their sets of points in an algebraically closed 
field. This leads to convoluted arguments, and (sometimes) theorems that are 
actually wrong. 

Example. Consider $G=\GL_n(k)$. For $g=(g_{i j})\in \matrices_n(k)$, we have 
$g\in G$ if and only if $\det(g)\ne 0$. But this isn't an honest algebraic 
equation. We can remedy this by noting that $\det(g)\ne 0$ if and only if 
there exists $y\in k$ such that $\det(g)\cdot y=1$. Thus we can define the 
\emph{coordinate ring} $k[G]$ of $G$, as 
\[
  k[G] = k[x_{i j},y] / (\det(x_{i j})y=1) .
\]
For $k$-algebras $A,B$, write $\hom_k(A,B)$ for the set of $k$-algebra 
homomorphisms $A\to B$. There is a natural identification 
\[
  \hom_k(k[G],k) = \GL_n(k) .
\]
For $\varphi:k[G]\to k$, put $g_{i j} = \varphi(x_{i j})$ and 
$b=\varphi(y)$. Then $\varphi$ is well-defined exactly if 
$\det(g)\cdot b=1$. So $\varphi$ is uniquely determined by the choice of an 
invertible matrix $g\in \GL_n(k)$. Since $b$ is determined by $g$ and $g$ can 
be chosen arbitrarily in $\GL_n(k)$, this correspondence is a bijection. So in 
some sense, $k[G]$ ``recovers'' $\GL_n(k)$. 

Now let $k$ be an arbitrary field. For concreteness, you could think of one of 
$\{\dR,\dC,\dQ,\dF_p\}$. Put $A=k[x_{i j},y]/(\det(x_{i j}) y-1)$, and define 
$\GL_n(R)=\hom_k(A,R)$ for any $k$-algebra $R$. We will think of 
``$\GL(n)_{/k}$'' as $\spectrum(A)$, which is a topological space with structure 
sheaf (essentially) $A$. The punchline is that the coordinate ring 
$k[G]$ of an algebraic group $G$ determines ``everything we need to know'' about 
$G$. 

Example. Let $k$ be a field not of characteristic $2$. Fix $d\in k^\times$. 
Let $G_d\subset \dA_{/k}^2$ be the subscheme cut out by $\{x^2-d y^2=1\}$. In 
other words, $k[G_d]=k[x,y]/(x^2-d y^2-1)$. We 
would like to realize $G_d$ as a matrix group. Consider the map 
$\varphi_d:G_d\to \GL(2)_{/k}$ given by $(x,y)\mapsto \begin{pmatrix} x & d y \\ y & x\end{pmatrix}$. 
This is an isomorphism between $G_d$ and the subgroup 
$\{g_{11}=g_{22},x_{12}=d x_{21}\}$ of $\GL(2)_{/k}$. 

[work this out as example of Weil restriction of $\Gm$ down from the ring of 
integers in $\dQ(\sqrt{-d})$.]

If $k=\bar k$, then consider the composite of $G_d\monic \GL(2)_{/k}\iso \GL(2)_{/k}$, 
the second map being given by 
\[
  g\mapsto \begin{pmatrix} & \sqrt d \\ 1 \end{pmatrix} g \begin{pmatrix} & \sqrt d \\ 1 \end{pmatrix}^{-1} .
\]
It sends $(x,y)\in G_d$ to $\begin{pmatrix} a & b\sqrt d \\ b\sqrt d & a \end{pmatrix}$. 
This has the same image as $\varphi_1:G_1\to \GL(2)_{/k}$. So if $k=\bar k$, then 
$G_d\simeq G_1$. 

Example. Set $k=\dR$. We claim that $G_1\not\simeq G_{-1}$. We give a topological 
proof. The group $G_1(\dR)\subset \dA^2(\dR)$ is cut out by $x^2-y^2=1$, hence 
non-compact. But $G_{-1}(\dR)=\{x^2+y^2=1\}$ is compact. Thus 
$G_1\not\simeq G_{-1}$ over $\dR$. 
But we have seen that $G_{-1}\simeq G_1$ ``over $\dC$.'' 

Exercise. Convince yourself that $G_1\simeq \Gm$. 

It turns out that ``twists of $\Gm$ over $k$ up to isomorphism'' are in 
bijective correspondence with $k^\times/2$, via the correspondence 
$d\mapsto G_d$. 





------------------------------

Let $G$ be linear algebraic group over $k$. There is a nice filtration of normal 
subgroups 
\[
  G\supset G^\circ\supset \rad G\supset \urad G \supset 1,
\]
where $G^\circ$ is the \emph{identity component}  for the Zariski topology) of $G$, 
$\rad G$ is the \emph{radical} of $G$, $\urad G$ is the \emph{unipotent 
radical} of $G$, and $1$ is the trivial group. 

The quotient $\pi_0(G)=G/G^\circ$ is a finite group, the quotient 
$G^\circ/\rad(G)$ is semisimple, the quotient $\rad G/\urad G$ is a torus, 
and $\urad G$ is unipotent. The group $\rad G$ is solvable. The quotient 
$G^\circ/\urad G$ is \emph{reductive}. 

[Draw diagram.]

We've seen examples of tori and unipotent groups, and everybody knows plenty of 
finite groups. 

Example. Semisimple group. Let $k=\dC$, and let $G$ be a connected linear 
algebraic group over $k$. We say that $G$ is \emph{simple} if it is 
non-commutative, and has no proper nontrivial closed normal subgroups. We say 
$G$ is \emph{almost simple} if the only such subgroups are finite. 

For example, $\SL(2)$ is almost simple, because its only nontrivial 
closed normal subgroup is $\{\pm 1\}$. 

A group $G$ is \emph{semisimple} if we have an isogeny 
$G_1\times \cdots \times G_r\to G$ with the $G_i$ almost simple. Here an 
\emph{isogeny} is a surjection with finite kernel. 

Over $\dC$, the almost simple groups (up to isogeny) are: 
\begin{center}
\begin{tabular}{c|c}
$A_n$ $(n\geqslant 1)$ & $\SL(n+1)$ \\
$B_n$ $(n\geqslant 2)$ & $\SO(2n+1)$ \\
$C_n$ $(n\geqslant 3)$ & $\Sp(2n)$ \\
$D_n$ $(n\geqslant 4)$ & $\SO(2n)$
\end{tabular}
\end{center}
We make requirements on the index in these families to prevent degenerate 
cases (e.g.~$B_1=1$) or matching (e.g.~$A_2=C_2$). There are five 
\emph{exceptional groups} 
\[
  E_6,E_7,E_8,F_4,G_2 .
\]
For example, $\dim(G_2)=14$ and $\dim(E_8)=248$. Later on, we'll be able to 
understand \emph{why} this list is complete. 
